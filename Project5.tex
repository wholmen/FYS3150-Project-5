\documentclass[a4paper, 12pt, titlepage]{article}

%
% Importering av pakker
%
\usepackage[latin1]{inputenc}
\usepackage[T1]{fontenc, url}
%\usepackage{babel}
\usepackage{textcomp}
\usepackage{amsmath, amssymb}
\usepackage{amsbsy, amsfonts}
\usepackage{graphicx, color}
\usepackage{parskip}
%
% Parametere for inkludering av kode fra fil
%
\usepackage{listings}
\lstset{language=python}
\lstset{basicstyle=\ttfamily\small}
\lstset{frame=single}
\lstset{keywordstyle=\color{red}\bfseries}
\lstset{commentstyle=\itshape\color{blue}}
\lstset{showspaces=false}
\lstset{showstringspaces=false}
\lstset{showtabs=false}
\lstset{breaklines}

%
% Layout
%
\topmargin = -40pt
\oddsidemargin = -15pt
\textheight = 710pt
\textwidth = 460pt

%
% Definering av egne kommandoer og miljøer
%
\newcommand{\dd}[1]{\ \text{d}#1}
\newcommand{\f}[2]{\frac{#1}{#2}} 
\newcommand{\beq}{\begin{equation*}}
\newcommand{\eeq}{\end{equation*}}
\newcommand{\bt}[1]{\textbf{#1}}
\newcommand{\n}{\newline}
%
% Navn og tittel
%
\author{Kandidatnummer }
\title{Project 5}
\title{FYS3150}

\begin{document}
 \maketitle
 \newpage
 
 \begin{section}*{Numerical solution to diffusion equation}
 \end{section} 
 
 \begin{section}*{1+1 dimensional problem, analytical solution}
  
 \end{section}
  
 \begin{section}*{1+1 dimensional, numerical notes}
  In this project I will solve the 1+1 dimensional diffusion equation by Monte Carlo methods using Markov 
  Chains. I will simulate the diffusion equation by programming the following random walk steps: \par
  1. Set initial number of particles, $N_0$, that start at $x=0$. This number will be conserved for $x = 0$
  throughout the whole simulation. \n
  2. Set up a vector, $u$, that contain positions of each individual particle. The positions will be split
  up into $Nx$ different bins with width $dx$. \n
  3. For each particle, draw a random number, $r$. If $ r < 0.5 $ then $ pos_{new} = pos_{old} - l_0$
     else $pos_{new} = pos_old + l_0$. \n
  4. If a particle moves beyond $x=1$, remove particle. \n
  5. If a particle moves from $x=0$, add a new at $x=0$ to maintain $N_0$. In addition: If the move is 
     negative, remove particle from vector. \n
  6. Repeat 2-5 for all time steps until final time is reached. \par
  I will implement two different algorithms. The first algorithm will have constant steplength
  \beq l_0 = \sqrt{2\Delta t} \eeq 
  and the other will have \beq l_0 = \sqrt{2\Delta t}\xi \eeq
  Where $\xi$ is a random number generated by a Gaussian distribution with mean value $0$ and standard
  deviation $\frac{1}{\sqrt{2}}$
 \end{section}
 
 
 \begin{section}*{Analytical solution of 2+1 dimensional problem} 
  We have seen before that the solution of a 1+1 dimensional problem is 
  \beq u(x,t) = \sum_{n=1}^{\infty} \f{-2}{\pi n} sin(n\pi x) e^{-n^2\pi ^2 t} + x - 1 \eeq
  To find the solution for the 2+1 dimensional problem, I will use separation of variables for both
  $X(x)$, $Y(y)$ and $T(t)$. I need boundary conditions for $X$ and $Y$ and initial condition for $T$.
  The boundary conditions for $X$ are given as \beq u(0,y,t) = 1, \;\;\; u(1,y,t) = 0 \eeq
  The initial condition is given as
  \beq u(x,y,0) = 0, \;\; 0 < x < 1 \eeq
  The boundary conditions for the Y-direction is not given, so I have to set them for myself. Since the 
  synaptic cleft is finite, the concentration of ions must be reduced in the y-direction, so a guess for 
  boundary conditions might be \beq u(x,0,t) = u(x,1,t) = 0 \eeq
  \end{section}

  \begin{section}*{2+1 dimensional problem. Numerical solution}
  the explicit method is stable for Nx = 10, T = 1 and Nt = 370. A lower dt will make the solution go crazy! 
  \end{section}

  
\end{document}
